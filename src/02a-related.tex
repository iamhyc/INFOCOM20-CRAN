\section{Background}
\label{sec:related}

\subsection{C-RAN Functional Split}
\comments{
    (Simply elaborate how and why we take such C-RAN implementation in 5G specification suggestion)
}

\subsection{Related Works}
%NOTE: MEC scheduling works
\comments{
    With the prosperity of large scale big data and computation intensive applications, mobile edge computing (MEC) is proposed to perform more real-time tasks than cloud computing \cite{mec-a1}.
    Edge computing aims at placing the computation task closer to user equipment (UE) and hence alleviate the communication overhead.
}

% There are several works considering but with simple CPU per cycles (CPS) consumption to evaluate the performance.
% In \cite{mec-b1}, the authors considered a mobile-edge computation offloading (MECO) system, both with local and edge computing.
% In \cite{mec-b1}, 
% In \cite{mec-b2}, 
% In \cite{mec-b3}, 
% In \cite{mec-b4}, 

%NOTE: vBBU study on channel multiplexing
% In \cite{cran-sim-PIMRC13}, the authors consider the signal processing of RRH offloading to the pooled baseband resource (vBBU), and analyzed the multiplexing gain by performing a simulation. The authors proposed a computation resource model whose Giga-Operations Per Second (GOPS) could scale with load and transmission mode. %each cell supports three sectors (three hexagonal area)

%NOTE: joint vBBU and vMEC scheduling
    